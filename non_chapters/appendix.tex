\begin{appendices}

\chapter{Summary}
Our genetic code is stored in our DNA, which consists of four bases: adenine (A), cytosine (C) guanine (G) and cytosine (C). More than three billion of these bases are strung together in 23 chromosomes, forming our genome. Each of our cells has two copies of our genome, containing instructions that guide cellular processes such as growth, development, signalling, and many more, but can occasionally also be the basis of disease. 

A large part of the genetic instructions are contained in our genes, which are small regions in our genome which contain information to make ribonucleic acid molecules (RNA). The making of RNA is also called gene expression. For some genes, which we call protein-coding genes, RNA can be translated by ribosomes to protein molecules, and proteins perform most of the functions in our cell. For example, proteins are used to digest food, to “communicate” between cells by sending and receiving signals, to create structures (e.g. bones, skin, hair, etc), and perform many other functions. Even the ribosome that translates RNA to proteins is itself a protein. For other genes, which we call non-coding genes, the RNA does not get translated to proteins, but the RNA molecule itself is used for regulation. For many of these non-coding genes the function is not known, although we know some are involved in regulating gene expression.

Although genes contain the information for functional molecules that regulate most of our cellular functions, together they only make up about 1\% of our genome. The other 99\%, once thought of as “junk” DNA with no function, is now known to be involved, among other functions, in regulation of gene expression. There are regions of the DNA that provide binding sites close to genes for the protein machinery that transcribes our DNA to RNA (promoter regions), some that provide protein binding sites further away from genes that help to activate the transcription (enhancer regions) or to silence it (silencer regions), or some are protein binding sites that insulate genes (enhancer-blocker insulators) or prevent structural changes in the DNA (barrier insulators). 

Variation in our genomes can change the regulation of genes or the functionality of the protein products. Most often, genetic variation is benign, having no (detrimental) effect. However, in some cases it can be the basis of disease. For many diseases we know from previous studies which regions in the genome can increase risk of a disease. For example, we know that having the DNA base G on position 44,892,362 of chromosome 19 confers a slight increase in risk of getting Alzheimer’s disease. However, it is often not known why this occurs. To get a better understanding of the molecular consequences of variations in the genome that might lead to disease, we want to know the downstream effects, and one such effect of interest is the change it can bring in the expression levels of genes.

This thesis describes how variation in our genomes affects gene regulation in blood and in the brain. Chapter 1 shortly introduces the concept of genetic variation and its effects on gene expression. In chapter 2 we use existing studies to look at the genetic architecture of molecular traits. We find that the effect of single nucleotide polymorphisms (SNPs; a variation of a single base) on disease is much more similar to the effect of SNPs on distal gene expression and protein levels than on nearby gene expression and protein levels. This suggests that SNPs regulating genes that are far away or on other chromosomes are more useful for understanding genetic causes of disease.

The previously published studies that we used focused on common variants. We wanted to know the effect of rare variants (variants that less than five in every thousand people have) on gene expression. To do this, in chapter 3 we used expression data from blood samples of 3,818 individuals to first identify genetic variation that was only present in a few individuals, and to then test if this variation has an effect on gene expression. We identified genes where this was the case, and also found that the effect on gene expression could be compensated for by overexpression of the second copy of the gene.

In the second part of this thesis, we look at genetic regulation of gene expression in various cellular contexts. Although each of our cells contain the same DNA, gene expression will not be the same between tissues and cell types. Partly, this is because genetic regulation of gene expression differs between cell types. In chapter 4 we look at differences in genetic regulation between blood cell types. Most gene expression is measured in samples that contain a mix of cell types (bulk samples) and the effect of specific cell types cannot be measured directly from these bulk samples. Because of this, we developed a computational method to predict cell count proportions and subsequently use the cell count proportions to model genetic regulation in separate cell types. We showed that this tool can accurately determine which cell types contribute most to the genetic signal we measure in the bulk samples. 

In chapter 5 we re-use and combine data from 14 published brain datasets to create a map of genetic regulation of gene expression in the brain. We applied statistical methods to this data to identify genes that are likely causal for various neurological disorders. To give one example, we identified CYP24A1 to be putatively causal for multiple sclerosis, and using our method developed in chapter 4 we show that it likely acts in neurons. Furthemore, we compared the genetic regulation in different regions of the brain with each other, and found that while most brain regions show similar regulation, the cerebellum is quite different. Finally, we created a co-expression network (a network of genes that go up or down in expression together) and used this as an additional way to prioritize likely causal disease genes.

Finally, chapter 6 summarizes the work done in this thesis and proposes ways in which future experiments can improve on this work. 




\chapter{Samenvatting}
Onze genetische code is opgeslagen in ons DNA, dat uit vier basen bestaat: adenine (A), cytosine (C), guanine (G) en cytosine (C). Meer dan drie miljard van deze basen zijn aan elkaar geregen in 23 chromosomen en vormen ons genoom. Elk van onze cellen heeft twee kopieën van ons genoom, met instructies die cellulaire processen sturen, zoals groei, ontwikkeling, communicatie en nog veel meer, maar kunnen af en toe ook de basis zijn van ziekte.

Een groot deel van de genetische instructies zit in onze genen, dit zijn kleine gebieden in ons genoom die informatie bevatten om ribonucleïnezuurmoleculen (RNA) te maken. Het maken van RNA wordt ook wel genexpressie genoemd. Voor sommige genen, die we eiwitcoderende genen noemen, kan RNA door ribosomen worden vertaald naar eiwitmoleculen, en eiwitten vervullen de meeste functies in onze cel. Eiwitten worden bijvoorbeeld gebruikt om voedsel te verteren, om te "communiceren" tussen cellen door signalen te verzenden en te ontvangen, om structuren te creëren (bijvoorbeeld botten, huid, haar, enz.) en om vele andere functies uit te voeren. Zelfs het ribosoom dat RNA afleest en eiwitten maakt is zelf een eiwit. Voor andere genen, die we niet-coderende genen noemen, wordt het RNA niet vertaald naar eiwitten, maar wordt het RNA-molecuul zelf gebruikt voor regulatie. Voor veel van deze niet-coderende genen is de functie niet bekend, hoewel we weten dat sommige betrokken zijn bij het reguleren van genexpressie.

Hoewel genen de informatie bevatten voor functionele moleculen die de meeste van onze cellulaire functies reguleren, vormen ze samen slechts ongeveer 1\% van ons genoom. Van de andere 99\%, ooit beschouwd als "junk"-DNA zonder functie, is nu bekend dat ze, naast andere functies, betrokken zijn bij de regulatie van genexpressie. Er zijn regio's van het DNA die bindingsplaatsen bieden dicht bij genen voor de eiwitmachinerie die ons DNA naar RNA transcribeert (promotorregio's), regio’s die eiwitbindingsplaatsen bieden die verder weg liggen van genen die helpen om de transcriptie te activeren (enhancer-regio's) of om het tot zwijgen brengen (geluiddempergebieden), en regio’s met eiwitbindingsplaatsen die genen isoleren (enhancer-blocker-isolatoren) of structurele veranderingen in het DNA voorkomen (barrière-isolatoren).

Variatie in onze genomen kan de regulatie van genen of de functionaliteit van de eiwitproducten veranderen. Meestal is genetische variatie neutraal en heeft het geen (nadelig) effect. In sommige gevallen kan het echter de basis van ziekte zijn. Voor veel ziekten weten we uit eerdere onderzoeken welke gebieden in het genoom het risico op een ziekte kunnen vergroten. We weten bijvoorbeeld dat het hebben van de DNA-base G op positie 44.892.362 van chromosoom 19 een lichte toename van het risico op het krijgen van de ziekte van Alzheimer geeft. Het is echter vaak niet bekend waarom dit zo is. Om een beter begrip te krijgen van de moleculaire gevolgen van variaties in het genoom  die tot ziekte kunnen leiden, willen we de stroomafwaartse effecten kennen. Een van die effecten is de verandering die het kan brengen in de expressieniveaus van genen.

Dit proefschrift beschrijft hoe variatie in onze genomen de genregulatie in bloed en in de hersenen beïnvloedt. Hoofdstuk 1 introduceert kort het concept van genetische variatie en de effecten ervan op genexpressie. In hoofdstuk 2 gebruiken we bestaande studies om te kijken naar de genetische architectuur van moleculaire eigenschappen. We vinden dat het effect van single nucleotide polymorphisms (SNP's; een variatie van een enkele base) op ziekte veel meer lijkt op het effect van SNP's op distale genexpressie en eiwitniveaus dan op nabijgelegen genexpressie en eiwitniveaus. Dit suggereert dat SNP's die genen reguleren die ver weg zijn of op andere chromosomen, nuttiger zijn voor het begrijpen van genetische oorzaken van ziekten.

De eerder gepubliceerde onderzoeken die we gebruikten waren gericht op veelvoorkomende varianten. We wilden weten wat het effect is van zeldzame varianten (varianten die minder dan vijf op de duizend mensen hebben) op genexpressie. Om dit te doen, hebben we in hoofdstuk 3 expressiegegevens uit bloedmonsters van 3.818 individuen gebruikt om eerst genetische variatie te identificeren die slechts bij een paar individuen aanwezig was, en om vervolgens te testen of deze variatie een effect heeft op genexpressie. We identificeerden genen waar dit het geval was, en vonden ook dat het effect op genexpressie kon worden gecompenseerd door overexpressie van de tweede kopie van het gen.

In het tweede deel van dit proefschrift kijken we naar genetische regulatie van genexpressie in verschillende cellulaire contexten. Hoewel al onze cellen hetzelfde DNA bevatten, zal genexpressie niet hetzelfde zijn tussen weefsels en celtypes. Dit komt deels doordat de genetische regulatie van genexpressie verschilt tussen celtypes. In hoofdstuk 4 kijken we naar verschillen in genetische regulatie tussen bloedceltypes. De meeste genexpressie wordt gemeten in monsters die een mix van celtypen bevatten (bulkmonsters) en het effect van specifieke celtypen kan niet direct uit deze bulkmonsters worden gemeten. Daarom hebben we een computationele methode ontwikkeld om de proporties van cellen te voorspellen en vervolgens deze proporties te gebruiken om genetische regulatie in afzonderlijke celtypen te modelleren. We hebben laten zien dat deze methode nauwkeurig kan bepalen welke celtypen het meest bijdragen aan het genetische signaal dat we in de bulkmonsters meten.

In hoofdstuk 5 hergebruiken en combineren we data van 14 gepubliceerde hersendatasets ome een kaart te maken van genetische regulatie van genexpressie in de hersenen. We hebben statistische methoden op deze gegevens toegepast om genen te identificeren die waarschijnlijk oorzakelijk zijn voor verschillende neurologische aandoeningen. Om een voorbeeld te geven, we hebben vastgesteld dat CYP24A1 vermoedelijk oorzakelijk is voor multiple sclerose, en met behulp van onze methode die in hoofdstuk 4 is ontwikkeld, laten we zien dat het waarschijnlijk in neuronen werkt. Bovendien vergeleken we de genetische regulatie in verschillende hersengebieden met elkaar, en ontdekten dat hoewel de meeste hersenregio's vergelijkbare regulatie vertonen, de regulatie in cerebellum heel anders is. Ten slotte creëerden we een co-expressienetwerk (een netwerk van genen die samen omhoog of omlaag gaan in expressie) en gebruikten dit als een extra manier om prioriteit te geven aan waarschijnlijke causale ziektegenen.

Ten slotte vat hoofdstuk 6 het werk samen dat in dit proefschrift is gedaan en worden manieren voorgesteld waarop toekomstige experimenten dit werk kunnen verbeteren.

\chapter{Acknowledgements}

One of the best parts of getting a PhD is meeting and working with many amazing people. I have been lucky to have had many great co-authors, colleagues and friends, who have made my time in Groningen really great. The genetics department was a very supportive place to work and I want to thank everyone that I've met during my time there for their help, discussions, and chats.

Lude, I am vary grateful to have had you as my supervisor. You always had a solution to any problem and I'm happy that when things went wrong you stayed positive and focused on solutions. It has been very inspiring to work with you and I hope to be able to implement many of the things I learned from you in the future. Also, thanks for giving me a lot of opportunities such as the summer schools I was able to go to and being able to visit Biogen in Boston.  \\
Cisca, it has been great to have worked in your department, I really enjoyed being able to have informal chats with you during the PhD lunches, and the critical feedback you gave during monday and tuesday meetings.

Harm-Jan, it has been really fantastic to work with you on MetaBrain. I have learned a ton from you and it was really fun, too. I liked our little beatboxing when we were working in the small meeting rooms. I admire how you are always helping everyone, and I hope we can do more research together in the future.

My paranymphs, Annique and Joana, thank you for being there for my defence. Annique, thanks for being a great friend and colleague1 It has made my PhD a lot more fun and easier to have you as a PhD buddy. I really enjoyed writing our paper in summer, and all our (spontaneous) nights out. Joana, my PhD time has definitely been a lot more fun with you as a friend. I'm happy that we now live relatively close by.

Freerk, although our project took much longer than we had hoped for, it's been great to work with you on it. I didn't know any bash before I started, so am happy I could learn it from you. I'm going to keep staying away from perl though. R\'aul, I never thought about colour schemes much before our deconvolution project, sadly our colour scheme alone was not enough to keep away the angry reviewer. Thanks for teaching me a ton of R skills. 

Everyone in Lude's group, I enjoyed all the discussions we have had, especially during lunchtime. Dylan, Sipko, and (although not in Lude's group) Kai, I really miss playing board games and hope can do that again some time. Urmo, we had a lot of nice dinners and drinks together, and I'm happy that I still see you whenever I visit Estonia. Adriaan, we always had good discussions over lunch, and hope to get some FLFL together again when I'm back in Groningen. Patrick, you always have a critical eye and I've really appreciated your feedback over the years. Juha, I was always impressed with your skills. It was great to see you in Boston, and I hope to see you in the future in other locations also. Anil, it was great meeting you in Switzerland, wish we had more time together in the same group. Monique, Alex, Shuang, Olivier, Tyler, Joeri, Floranne, Irene, Pauline, Floranne, Roy, Robert, Marc-Jan, Dasha and Harm, I have always worked with you with a lot of enjoyment and always felt supported in our group. Janneke, you have the most positive attitude, thank you for helping me with all the paperwork.


Jackie, thanks for editing chapter 2. Kate, thanks for editing all other chapters. I'm incredibly lucky that you were available to do this for me, as you greatly improved my writing. 

Yang, I really enjoyed working with you on deconvolution of eQTL effects. I learned a lot about linear models from you. Iris, it was fun working on the ATAC-seq data. Morris, thanks for the support and help with chapter 3. Jingyuan, Sasha, Serena, Vinod and Seb, thanks for making the department a supportive place to work, and your discussions during the lunch meetings.

Ellen, I miss our (bi)-weekly video calls! It has been really great collaborating with you, and thank you so much for making my visit to Boston smooth and fun. Heiko, thanks for your valuable inputs, and thank you for letting me visit your group, it was a really great experience. Denis, it was really nice spending time with you at ASHG. Jimmy, Chia-Yen, Paula, Yungfeng and others at Biogen, thank you for making me feel welcome (and thanks for lending me your bike key Chia-Yen).

Niek, we hebben alleen via e-mail aan verschillende EWAS samengewerkt, maar dat ging heel soepel en fijn. Hopelijk komt zo'n mogelijkheid nog eens. 

Zuzanna, you and Konrad were always so welcoming, and your apple pie drink during the ski vacation made me forget my knee pain.

Peter-Bram, Joost, Ramin, Szymon, it is quite some time ago that we were having our BIOS calls, but I enjoyed them. Thanks for your help with everything.

I have been lucky to have (co)-supervised some really great students. Anne, Carlos, Omar, and Martijn, it has been really fun to work with you, and you have all been a great help. 

The people of the GCC, without you I wouldn't have been able to complete this work. It was very fun to work in your corner. Pieter, you have helped and fixed my cluster problems innumerable amounts of times, and patiently answered many of my questions. Thanks a lot! Gerben, Martijn, and Roan, I have used a lot of your Molgenis Compute scripts. Marieke and Mariska, bedankt voor het helpen met data management. I would also like to thank the people from Molgenis, I really enjoyed joining for the koffer when I could. 

Sido, bedankt voor het helpen met het opzetten van docker en ansible en de duizenden vragen die ik erover had. 

Fleur and Jonatan, thanks for helping me with using molgenis through a python API early on. 

I would not have started doing a PhD without the mentors and teachers that I've had over the years. I would like to thank three specifically. Martijn, jou geweldige enthousiasme tijdens de intro dag was een grote reden dat ik met bioinformatica begonnen ben, en tijdens de studie heeft het ook mijn enthousiasme aangewakkerd. Sue, I got my first research experience with you and I think the half year I spend in your group has been the most educational half year I've ever had. Francisco, you gave me a lot independence on my project and send me to my first conference, I decided I wanted to do a PhD after my research with you.

The best group of people, Team Groningen. It was great to always have you around to party, go on trips, to festivals, playing games, 5 hour long brunches, and all the other experiences we shared. Tejas, Hannah, Joana, Adele, Vincent, Sharon, Alberto, Lucile, Filippo, Paolo, Annique, D\'esir\'ee, Anne-Grete, and Steven, my time in Gronignen wouldn't have been half as fun without all of you. 

Frits en Berdien, jullie hebben mij met open armen ontvangen in Groningen. Ik heb met ontzettend veel plezier in het tuinhuisje gewoond, en voelde mij altijd welkom.

Thank you Mitchell and Mayra for making me feel at home in Groningen when I didn't know anyone yet. You are both always super nice. I hope we can go to paradigm together sometime in the future again.

\"Aare and Tiiu, suur aitah for letting me write my introduction and discussion in your (country) house.

Lieve papa en mama, bedankt voor al jullie steun door de jaren heen. Het is ontzettend fijn dat jullie altijd in mij geloven en altijd gesteund hebben. Ook fijn dat jullie mij opgevangen hadden met mijn gebroken knie. Het is altijd fijn om weer even bij jullie thuis te zijn. Zonder deze steun zou dit boek er niet zijn geweest. Luuk, bedankt voor de vele leuke trips met wijn, duiken en andere leuke dingen die het leven beter maken, en dat ik altijd een plek bij jullie heb als ik in een sneeuwstorm beland. Jij bent mijn favoriete broer. And also thank you Lisa, making all those trips even more fun. 

Lieve Anne-grete, minu kallis, thank you for always being there for me. You made finishing this book easier and less stressful. Especially with finishing during corona, it was great to have you around and supporting me. You always kept me positive and smiling when I was frustrated. Thank you for everything. I'm very happy that I shared most of this journey with you and that we can defend in the same week!

\chapter{About the author}

Niek Peter de Klein was born on the 9th of December, 1990, in Velp, the Netherlands. He completed his bachelor degree in Bioinformatics at the Hogeschool van Arnhem en Nijmegen, followed by a Masters degree in Bioinformatics at the Vrije Universiteit in Amsterdam. During his Bachelors and Masters degree, he got research experience in labs in Stanford, Dundee,Amsterdam, Luxembourg, Dundee. His research topics were on gene expression regulation and method development for mass spectrometry analysis.  

In 2015, Niek started his PhD at the Genetics Department of the University Medical Center in Groningen, the Netherlands in the group of prof. Lude Franke. His research focussed on genetic regulation of gene expression in different tissues. 


Currently, Niek is working as a postdoctoral researcher at the Wellcome Sanger Institute in Hinxton, UK. He is working in the labs of Dr. Emma Davenport and Dr. Gosia Trynka on identifying likely causal genes of systemic lupus erythematosus that might be used as future drug targets.



\renewcommand{\bibsection}{\section*{List of Publications}}
\nocite{*}
\bibliographystyle{habbrvyr}
\bibliography{non_chapters/appendix}


* Contributed equally

\end{appendices}

\chapter{Introduction}
\label{chap:introduction}

{ \Large \leftwatermark{
\put(-76.5,-75){\includegraphics[scale=0.8]{img/thumbindex.eps}}  \put(-67,-66.5){ {\color{white} 1 }}
\put(-67,-91.5){ 2 }
\put(-67,-116.5){ 3 }
\put(-67,-141.5){ 4 }
\put(-67,-166.5){ 5 }
\put(-67,-191.5){ 6 }
\put(-67,-216.5){ 7 }
\put(-67,-241.5){ 8 }
} \rightwatermark{
\put(346.5,-75){\includegraphics[scale=0.8]{img/thumbindex.eps}}  \put(350.5,-66.5){ {\color{white} 1 }}
\put(350.5,-91.5){ 2 }
\put(350.5,-116.5){ 3 }
\put(350.5,-141.5){ 4 }
\put(350.5,-166.5){ 5 }
\put(350.5,-191.5){ 6 }
\put(350.5,-216.5){ 7 }
\put(350.5,-241.5){ 8 }
}}

\newpage

\noindent
Lorem ipsum

\section[Table contents title]{Table contents title: full title} \label{intro_origin}

\begin{table}
\begin{tabulary}{\linewidth}{LL}
  \mbox{Term~~~~~~~~~~~~~~~~~~~~~~~~~~~~~~~~~~~~} & Definition \\
  \hline
  \rule{0pt}{2.5ex}Allele & A variant form of a gene or genetic locus \\
  \rule{0pt}{2.5ex}Amino acid & Small organic building block of proteins \\
  \rule{0pt}{2.5ex}Base & Building block of nucleic acid \\
  \rule{0pt}{2.5ex}Base pair & Two bases bound by hydrogen in the DNA double helix \\
  \rule{0pt}{2.5ex}Chromosome & Organizational unit of DNA, humans have 22 pairs plus XX or XY \\
  \rule{0pt}{2.5ex}Codon triplet & Sequence of three bases that codes for a specific amino acid \\
  \rule{0pt}{2.5ex}Complex disease & A disease caused by the joined effect of multiple environmental and genetic factors \\
  \rule{0pt}{2.5ex}Conserved loci & Genomic locations that have changed little in evolution \\
  \rule{0pt}{2.5ex}Diagnostic yield & The percentage of solved patient cases \\
  \rule{0pt}{2.5ex}DNA & Deoxyribonucleic acid, encodes the genetic information of an organism \\
  \rule{0pt}{2.5ex}Dominant disease & A disease caused by a single pathogenic allele on one chromosome of a pair \\
  \rule{0pt}{2.5ex}Enzyme synthesis & Production of proteins that act in, or execute chemical reactions \\
  \rule{0pt}{2.5ex}Exon & Short for 'expressed region', coding sections of a DNA sequence \\
  \rule{0pt}{2.5ex}Genetic code & Rules by which nucleic acid is translated into messenger RNA \\
  \rule{0pt}{2.5ex}Genetic inheritance & Transmission of inborn traits from parent to offspring \\
  \rule{0pt}{2.5ex}Genome sequencing & Determining the order of bases in a genome \\
  \hline
\end{tabulary}
\caption[Glossary of key terms, pt. 1/2.]{\label{table:introduction_glossary_1} Glossary of key terms used in this introduction, pt. 1/2.}
\end{table}


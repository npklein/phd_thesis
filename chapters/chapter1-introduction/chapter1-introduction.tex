\chapter{Introduction}
\label{chap:introduction}

{ \Large \leftwatermark{
\put(-76.5,-75){\includegraphics[scale=0.8]{img/thumbindex.eps}}  \put(-67,-66.5){ {\color{white} 1 }}
\put(-67,-91.5){ 2 }
\put(-67,-116.5){ 3 }
\put(-67,-141.5){ 4 }
\put(-67,-166.5){ 5 }
\put(-67,-191.5){ 6 }
\put(-67,-216.5){ 7 }
\put(-67,-241.5){ 8 }
} \rightwatermark{
\put(346.5,-75){\includegraphics[scale=0.8]{img/thumbindex.eps}}  \put(350.5,-66.5){ {\color{white} 1 }}
\put(350.5,-91.5){ 2 }
\put(350.5,-116.5){ 3 }
\put(350.5,-141.5){ 4 }
\put(350.5,-166.5){ 5 }
\put(350.5,-191.5){ 6 }
\put(350.5,-216.5){ 7 }
\put(350.5,-241.5){ 8 }
}}

\newpage

\noindent

\section{General introduction}
\subsection{Overview of the use of genetics and genomics in healthcare}
\subsubsection{Motivation of this thesis}
When the first draft of the human genome was released in June of 2000 the then president of the United States, Bill Clinton, told the press that the genome sequence would "revolutionize the diagnosis, prevention, and treatment of most, if not all, human diseases".\cite{collinsHasRevolutionArrived2010a} It is that this has revolutionized genomic research. Up until that time rare monogenic diseases where studied most, often with time-consuming methods. After the first draft of the human genome was completed, there was a four-fold increase in causal genes identified for rare, single-gene disorders. \cite{claussnitzerBriefHistoryHuman2020b} Additionally, computational and DNA sequencing technologies that were invented or optimized for this project\cite{hoodHumanGenomeProject2013} have been further developed and helped bring down the cost of DNA sequencing dramatically, from over \$2 billion for the first human genome to less than \$1000 now. It has made it possible for large projects like HapMap\cite{InternationalHapMapProject2003} and ENCODE\cite{consortiumENCODEENCyclopediaDNA2004} to produce maps of sequence variation and identify functional elements in the human genome. \\

However, although the impact of the project was and still is huge and has allowed for many advancements within genetics and genomics research\cite{chicheBenchtobedsideReviewFulfilling2002a} and identification of disease genes and genetic diagnosis\cite{claussnitzerBriefHistoryHuman2020b}, it has only provided treatment in limited cases. Additionally, even though diagnosis of genetic disorders has improved, for the majority of cases genetic testing is still unable to find the causal genes and variants\cite{diemenRapidTargetedGenomics2017a}. Genetic tests are successfully used for preventative care, such as mastectomies or extra surveillance for BRCA1/2 carriers \cite{heemskerk-gerritsenSurvivalBilateralRiskreducing2019a}, but for complex disorders, disorders where many variants have a small effect as opposed to 1 or a few variants having a large effect, this is still difficult\cite{claussnitzerBriefHistoryHuman2020b}.  \\

So even though the past 20 years has been full of incredible advances in genomics and genetics research, we are not even close to the prevention and treatment of most human diseases, nor do I think that this will happen in the next 20 years. I do think that the field of genetics is now at a great position to start assisting in complex diseases as well as drug development. There are a lot of public-private partnerships under way that will help in a better understanding of genotype-phenotype relationships\cite{HomepageInternationalCommon}, provide large data resources for researchers to use\cite{finngenFinnGenDocumentationR2,sudlowUKBiobankOpen2015}, or assist in drug target identification\cite{arvalho-silvaOpenTargetsPlatform2019}. Improvements in polygenic risk scores will allow prediction of risk for polygenic traits\cite{kheraGenomewidePolygenicScores2018}, which can help in preventative care. In drug development the drug compounds with direct genetic support increase substantially between preclinical and approved stage (2.0\% and 8.2\% respectively\cite{nelsonSupportHumanGenetic2015b}), showing the added benefit of integrating genetic research in the drug development pipeline. Finally, integrating layers of omics and environmental data will address the missing heritability \cite{claussnitzerBriefHistoryHuman2020b}. \\

In this thesis I address two important factors needed for using genetics to assist in approving health care. A better fundamental understanding of 
	

\subsubsection{something about current methods for complex disorders}
To help although polygenic risk scores are sometimes used\cite{natarajanpradeepPolygenicRiskScore2017}.

\subsubsection{Role of gene regulation}
Almost everything in your cells is regulated by genes and their transcriptional products. The protein product of protein coding genes act as transcription factors (regulators of gene activity), antibodies (binding to specific foreign particles), enzymes (enablers of many chemical reactions), messenger, structural component or for transport or storage \cite{uzmanMolecularBiologyCell2003}, look up relevant page numbers>, while the transcripts of non-coding genes are often involved in regulating the activity of genes \cite{shabalinaMammalianTranscriptomeFunction2004a}. Gene activity, as measured by the number of functional transcripts that are produced, is dependent on genetics. For example, a change of a single nucleotide in the DNA (Single Nuclear Polymorphism, SNP) at a transcription start site can cause a transcription factor to bind with less affinity, thereby lowering gene activity. To understand how the cell functions we therefore have to understand how genetics is influencing the genes activity. 

\subsubsection{Role of SNPs}
The field of genetics has focussed on identifying DNA sequence variants that influence biomedical traits, especially of human diseases\cite{claussnitzerBriefHistoryHuman2020b}. Thanks to large collaborative efforts there are now biobanks with genotype data for more than 100.000 indivuals (UK biobank <citation>, FinnGenn <citation>, Estonian Genome Project <citation>, more?). With these large datasets it is possible to identify those DNA sequence variants that increase or decrease risk for many different (disease) phenotypes, and as of <date> the GWAS Catalog <citation> contains <number> of risk factors for <number> of phenotypes. 

\subsubsection{Advances in technology}
Thanks to advances in technology over the past decades it is now possible to get both the DNA sequences as well as gene expression levels for multiple thousands of individuals. We can compare if people that have a specific SNP have a difference in gene activity than people that do not have that SNP. If this SNP is also known to be present more often in individuals with a certain disease, we can get a better understanding of the disease by studying the gene that is affected by the SNP. 

\subsubsection{Difference complex / mendelian diseases and traits}

\subsection{Human genetic evidence for drug development}
One area where genetics can be of definite help is in the pre-selection of candidate drug compounds. It is very costly to perform clinical trials, and even though there have been many technological advances in recent decades, drug development costs are high while there has been little increase in the output of new drugs \cite{cookLessonsLearnedFate2014a}. By including genetic information in the drug compound selection the succcess rate in clinical development can be doubled\cite{nelsonSupportHumanGenetic2015b}. 

\section{This thesis}
The aim of this thesis is to get a better understanding of the context in which genetic risk factors influence gene expression levels. In part 1 we describe the effects of rare variants on gene expression in blood and the genetic architecture of molecular traits. In part 2 we identify the genetic regulation in context of different cell types in blood and in brain.

\subsection{Part 1 - architecture of molecular traits and rare variant signal}
In Chapter 2 we compared the heritability explained by all studies included in the GWAS catalog <citation> to the heritability that is explained by expression quantitative trait loci (eQTL) and methylation quantitative trait loci (meQTL) studies. We found that the genetic architecture of molecular traits is less polygenic than that of (disease) phenotypes. In chapter 3 we identified the disease categories for which allele-specific expression can be informative for prioritizing putative causal variants.

\subsection{Part 2 - cell type and tissue context of genetic regulation of gene expression}
In chapter 4 we developed two methods: 1: predicting the cell count proportions from whole blood samples using gene expression data, and 2: deconvoluting the cell type interacting eQTL effects found in whole blood samples. In chapter 5 we compared the genetic regulation of gene expression between brain and blood and identified which genes are affected by neurodegenerative and psychiatric risk factors.


\bibliographystyle{naturemag}
\bibliography{chapters/chapter1-introduction/chapter1-introduction}
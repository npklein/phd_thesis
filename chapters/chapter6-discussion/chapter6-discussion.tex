\chapter{Discussion and Perspectives}
\label{chap:discussion}

{ \Large \leftwatermark{
\put(-67,-66.5){ 1 }
\put(-67,-91.5){ 2 }
\put(-67,-116.5){ 3 }
\put(-67,-141.5){ 4 }
\put(-67,-166.5){ 5 }
\put(-76.5,-199){\includegraphics[scale=0.8]{img/thumbindex/thumbindex_wongOrange.eps}} \put(-67,-191.5){ {\color{white} 6 }}
} \rightwatermark{
\put(350.5,-66.5){ 1 }
\put(350.5,-91.5){ 2 }
\put(350.5,-116.5){ 3 }
\put(350.5,-141.5){ 4 }
\put(350.5,-166.5){ 5 }
\put(346.5,-199){\includegraphics[scale=0.8]{img/thumbindex/thumbindex_wongOrange.eps}} \put(350.5,-191.5){ {\color{white} 6 }}
}}

\newpage

\section{Summary of this thesis}
The regulatory networks in which genes form pathways are incredibly complex, with many layers of regulation. When trying to understand the function of a gene or the potential of a gene to be a candidate drug target, it is important to know how this gene functions within these complicated pathways. One way to approach this is by identifying which genetic factors affect the genes through eQTL studies, and subsequently building gene correlation networks to identify which pathways these genes are involved in. Preferably this should be done in cell type or tissue that is relevant for the phenotype of interest. In this thesis we have outlined the genetic architecture of molecular traits in Chapter 1\cite{claringbouldGeneticArchitectureMolecular2017} and identified the effect of rare variants on gene expression data in whole blood in Chapter 2. We have developed statistical methods to look at cell type interacting genetic effects in blood in Chapter 4\cite{raulaguirre-gamboaDeconvolutionBulkBlood2020}, and we have analysed the effects of common variants on gene expression in the brain in Chapter 5. I will now discuss how this data and these methods can be used to help prioritize candidate causal gene prioritization, assist in assessing the feasibility of drug compounds and create hypotheses for the role of identified genes on disease phenotypes. Finally, I will give a perspective of what the future direction of this work could be. 

\section{Importance of this work}
Like much of the computational work in Genetics, the goal of the work in this thesis is being able to better prioritize genes.

\subsection{Tissue and cell type specificity}
From previous studies, such as GTEx\cite{lonsdaleGenotypeTissueExpressionGTEx2013}

\section{Future perspective}
In this thesis I have shown some techniques and genotype-phenotype maps that can assist in better the understanding of health and disease. 

\subsection{Single cell RNAseq}
\subsection{Phenotype genotype maps}
\subsection{Drug predictions}
One thing not addressed in this thesis is the prioritisation of drug compounds using genomics information. One of the tools that can be used is the connectivity map\cite{lambConnectivityMapUsing2006}. It has measured gene expression levels in cell cultures that were treated by different drug compounds, and compared that to gene expression levels of cell cultures that were not treated by any drug compounds. Fisetin\cite{gibbsHumanGenomeProject2020}. 